\documentclass{article}
\usepackage[T1]{fontenc}
\usepackage{textcomp}

%%  Latex generated from POD in document benchmark-blender.pl
%%  Using the perl module Pod::LaTeX
%%  Converted on Tue Feb 24 18:54:15 2015


\usepackage{makeidx}
\makeindex


\begin{document}

%% \tableofcontents

\section{PROGRAM benchark-blender.pl\label{PROGRAM_benchark-blender_pl}\index{PROGRAM benchark-blender.pl}}
\begin{verbatim}
        Blender Automated Benchmark Suite (BABS)
        [cc-by] Piotr Arlukowicz, <piotao@polskikursblendera.pl>
\end{verbatim}


Documentation: read at least config section below :)

\subsection*{USAGE\label{USAGE}\index{USAGE}}
\begin{verbatim}
        benchmark-blender.pl [options] -b blendfile
        benchmark-blender.pl -h
\end{verbatim}
\subsection*{OPTIONS\label{OPTIONS}\index{OPTIONS}}


Options are specified via command line using a typical getopt notation. Each
option should start with \texttt{-} and is one letter long. Do not join options with
their parameters, always add spaces around. Bad: -i2, good: -i 2.

\begin{verbatim}
        -e  file       - blender executable (default: blender)
\end{verbatim}


The blender executable file itself. In Linux, this will be \texttt{blender}, while in
windows this can be \texttt{blender.exe} or \texttt{blender.lnk}.

\begin{verbatim}
        -p  path       - blender path (default: .)
\end{verbatim}


Path to the blender. Can be absolute or relative to the place where you are
run the script. Usually I keep Blender in quite a long path, and this could
be a typical case. The location of: \texttt{/home/piotao/bin/blender-git/install/linux}
can be expressed by several shorter terms: \texttt{\%home/\%git/\%install}. Then, if
you are using this terms, you have to put them into CONFIG section inside the
script, like this:

\begin{verbatim}
        %home => '/home/piotao',
        %git  => 'bin/blender-git',
        %install => 'install/linux',
\end{verbatim}


The script will substitute all config variables from those information.
In this way you can make shortcuts and aliases, and run some additional
commands along with blender testing.

\begin{verbatim}
        -b  file       - blender file (HAS to be specified)
\end{verbatim}


Normal BLEND file to be used for testing. This file will be run through
blender with options given in another sections of config (or by \texttt{-c} option).
Usually this has to be specified, or written into config section.

\begin{verbatim}
        -c  string     - the entire command to run blender with options
\end{verbatim}


This is the command the script executes. The output of this command has to have
a string: \texttt{Time: (xx:yy.zz)}. The script can catch this time and collect all
results for each rendered frame and calculate the average time. If you have to
run Blender for an animation, and you want to catch times of render per frame,
you have to run Blender as a standalone program, and then you have to catch the
log. The info about time is in this log. You can then put this log to be parsed
by this script with \texttt{-l} option.



The typical command is by default composed from another config parts, for
example:

\begin{verbatim}
        blenderCommand => '%blenderPath/%blenderExe %options',
\end{verbatim}


The script is building the right command from this \%-something pieces. Each of
them has to be specified in config sections, and you can have your own defs and
strings defined there too. All tokens with \texttt{\%} sign are replaced by their
values defined somewhere in another parts of the config, so finally you can get
the whole command. If there is a recursion problem, script will catch this and
stop.  So, if you define \texttt{\%blenderPath} and \texttt{\%blenderExe} like this:

\begin{verbatim}
        blenderPath => '/bin',
        blenderExe  => 'blender_x86',
\end{verbatim}


Then, if you have your options like this:

\begin{verbatim}
        options  => '%options2',
        options1 => '-noaudio -noglsl -nojoystick ',
        options2 => '%options1 -b %blenderFile %thr -f 1 2>&1 ',
\end{verbatim}


Then this will give you the whole options like build as:

\begin{verbatim}
        options=>'-noaudio -noglsl -nojoystick -b %blenderFile %thr -f1 2>&1 '
\end{verbatim}


...which will then have the \texttt{\%blenderFile} and \texttt{\%thr} token replaced by their
corresponding values. For blender file you have to specify \texttt{-b} option,
and for threads number to use you have to put \texttt{-t} option.

\begin{verbatim}
        -a  float      - maximal change of averaged time
\end{verbatim}


This is the number, which stands for precision of testing. Each new test which
was carried out by running blender is measured in terms of time. This measured
time is different (probably) from each other time, so it is averaged, to gain
the best approximation.  But, how you can tell the average time is good enough?
How many runs you really have to do, in order to get the good average time?
Please notice, that running blender three, or even ten times, and averaging all
times is not enough, if your times are very different from each other. So, this
value can force program to run tests ENOUGH times, and after each time the new
average is calculated from all gained tests. This new average is compared to
the old one, and IF the old one differs no more than this value, then all tests
are finished. To understand this, consider this example:



Here, we have just a few times measured during some blender runs. They have
diffrent values. How many of such runs is enough to have a good average value
which do not change much with next tests? Let's assume we can accept 0.1
change between two consecutive averaged results.

\begin{verbatim}
        No Time measurements            Average of times so far         Diffrence of average
  1   10                  10 (of course!)           (nothing yet)
  2   11                  10.50                     0.50
  3   10                  10.33                     0.16
  4   12                  10.75                     0.42
  5   10                  10.60                     0.15
  6   10                  10.50                     0.1   -> this is the point
  7   10                  10.43                     0.07
  8   10                  10.38                     0.05
  9
\end{verbatim}


As you can see, during tests, we renderd 8 times, and the results were
collected. For each new result, we calculated average value of all measurements
and this new average value is put into the third column. Now, we can calculate
the diffrence between our new average and the previous one. This value is our
\textit{epsilon} - the change which tells us, how much the average is changed between
tests. If this change is small enough, we can stop further testing, because
our result will be no better that current one! So, in this example we know,
that there is only 6 run sufficient instead of 8, which can save some time. In
fact, this procedure is executed as long as the change of averages is smaller
than the epsilon. In our case we assumed that we are acceptig 0.1 value, so
change smaller that that do not count much. This value is set by default to
0.01 (you have to specify a float number, not percent).



For lower epsilon values the computations can lasts longer and some of them can
take a really long time. On my 6-cores cpu, under linux, where only 10-20 
processes were run in parallel (even cron was killed tho), this procedure has
taken 47 times, until the level of 0.01 was reached. For the same computer, in
the save environment, the GPU has to use only 8-10 iterations.

\begin{verbatim}
        -i  int        - additional runs after time is stable
\end{verbatim}


This is the number which tells how many additional runs the script should run
AFTER the epsilon limit is reached. Look at the above example: we got the
demanded minimum value in 6 iteration. Instead of stopping the calculation
process, there were two additional runs performed to ensure the stability of 
average value. It seems that this is necessary, because there are some random
fluctuations and some tests can raise the average above the limit. If that is
the case, the value of repetitions is reset, and after gaining the minimum
again, this number of additional runs still has to be performed.

\begin{verbatim}
        -t  int        - number of threads blender use
\end{verbatim}


This option is used to specify how many threads should blender use in order to
render on CPU. For GPU this value is ignored. If this number is set to 0, which
is the default, then all threads are used in auto-detect mode and blender will
decide itself. If a value is given here, then this option appears in command
line and specifies the desired number of threads to use, by applying -t NUM
option to blender command line.

\begin{verbatim}
        -h, --help     - prints this help and exits.
\end{verbatim}


This is simple: it displays a short help about the program.

\begin{verbatim}
        -v             - verbose output and some debug
\end{verbatim}


This option helped me to design this program and debug it. You do not have to
use it, and I leaved it here just for those, who are curious :)

\begin{verbatim}
        -l  txtfile    - blender log file to analyze only
\end{verbatim}


If you do not want to run blender with any blender file, but you have a log
output from some runs, you can just analyze that log file with no blender at
all. To generate results in the same format like those generated during a real
tests, specify just a single \texttt{-l logname} option, where \texttt{logname} is just a
plan text file. This text file should be the complete and exact log file right
from blender. It can contain multiple frames renderings, etc. but only time
and last read blend file are catched. Only one log file is processed at a time
and it has a precedence over blend file and blender run.

\subsection*{CONFIG SECTION\label{CONFIG_SECTION}\index{CONFIG SECTION}}


In the config section you can define all configuration you need to run this
script without any parameters. However, configuration have it's own limitations
so do not expect miracles :) This is perl, and I made this script flexible
enough to suit well my needs. So, here it is.



All config definition is build as KEY =$>$ VALUE. KEY has to be unique string.
VALUE can be a number, a string, etc. It can be even another structure, but
then you have to know what you are doing (learn Perl! - this is very good
language).  So, if you define something like:

\begin{verbatim}
        script => 'perl',
\end{verbatim}


notice that what is on the right side HAS to be in apostrophes. The left side
is automatically cited, so no apostrophes are needed. However, if you like this:

\begin{verbatim}
        big idea => 'my mind',
\end{verbatim}


This will not work, because KEY has to be a single string, so you have to cite
it this way:

\begin{verbatim}
        'big idea' => 'my mind',
\end{verbatim}


This will not work either, because keys with spaces won't be parsed correctly.
When you have some such keys and values defined, you can use them to define new 
values. Note however that you can't use them as mixed key names. So this works:

\begin{verbatim}
        newidea => 'this is %script',
\end{verbatim}


but this doesn't work:

\begin{verbatim}
        %script => 'something',
\end{verbatim}


Note that newidea key will be translated to 'this is something' because of
substitutions.  So you can make some even more, nested substitutions, and they
are handy if you have to keep diffrent configurations inside one script. Look
at this:

\begin{verbatim}
        blenderCommand => '%setup1',
\end{verbatim}
\begin{verbatim}
        setup1 => '.....',
        setup2 => '.....',
\end{verbatim}


In this case one can change only one option at the start of the script, and all
is changed respectively. One can then specify new configuration option via the
command line, putting \texttt{-c '\%setup2'} parameter, and this will use a whole new,
and different definition from the script config. I think this is so handy, that
should suffice my needs and maybe even yours.



Remember that some of keys in the config file are reseved and they are used in
the code directly (however this may change in the Future). There are:

\begin{verbatim}
        blenderCommand
        maxAverageChange
        maxRunsToEnsure
        verbose
        blenderFile
        version
\end{verbatim}


Also, the script will create some new config keys during its work, so they are
also kind-of reserved, because even if you defined them, they can be
overwritten. They are:

\begin{verbatim}
        framecount
                - this is the numer of tests performed so far
        start
                - time in seconds from EPOCH when script just started
        end
                - time in seconds when script is about finish
        totalsum
                - sum of all time measurements
        average
                - current average value
        frames
                - a list of all collected time measurements (and more)
\end{verbatim}


All other keys are not mandatory and you can change them, use them, etc.

\subsubsection*{Recursion\label{Recursion}\index{Recursion}}


One can mistakenly made a recursive definitions in config, for example:

\begin{verbatim}
        one => '%two',
        two => '%one',
\end{verbatim}


This naturally can lead to infinite recursion, but I thought a bit and coded
simple thing to prevent this. Script will, however, fail when such a recursive
definition is found. This applays also to cases where you can build token with
recursive calls which at some point can lead to situation where token should be
explained by itself. I'm not sure if my security code is enough, so better try
not to build recursive definitions.

\subsubsection*{Output of the blender command\label{Output_of_the_blender_command}\index{Output of the blender command}}


This script works with the assumption that blender will print a log of it's
rendering process to the standard output. Script is then reading this output
using normal redirection pipe. If you change this behaviour, the results can be
unpredictable. So, you have to assure that \texttt{blenderCommand} will be executing
a command (or commands) which are able to print to the standard output. The
text which is printed doesn't really matter - in the whole stream of log there
is only one string searched. This is Time: xx:yy.yy value. Blender prints this
time for both internal and Cycles renderers, if you run it with console output.
If you plan to include some intermediate scripts between this script and
blender, you should ensure that at least this line with the text Time ... is
properly outputted to the output stream.



Also, to avoid messy output, the \texttt{blenderCommand} is defined with a stderr
redirection to stdout. Thanks to this you don't have to see any warnings or
messy error messages printed by Blender itself. This of course hides some info
so you should test your blend files and render to see if everything works.

\subsection*{DEFAULT CONFIG\label{DEFAULT_CONFIG}\index{DEFAULT CONFIG}}


My default config section is here (in case you've lost it):

\begin{verbatim}
        maxAverageChange => '0.01',
        blenderExe  => 'blender',
        blenderPath => '%bin/%git/%install',
        blenderFile => 'any.blend',
        blenderCommand => '%blenderPath/%blenderExe %options2',
        options1 => '-noaudio -noglsl -nojoystick ',
        options2 => '%options1 -b %blenderFile %thr -f 1 2>&1 ',
        threads => 0,
        verbose => 0,
        version => '0.1',
        start   => time,
        home    => '/home/piotao',
        bin     => '%home/bin',
        git     => 'blender-git',
        install => 'install/linux',
\end{verbatim}
\subsection*{DEFAULT OPTIONS\label{DEFAULT_OPTIONS}\index{DEFAULT OPTIONS}}


If you would like to enter command line to redefine all config sections, here
is the complete set of options:

\begin{verbatim}
        -e 'blender'                      # or 'blender.exe'
        -p '/path/to/blender'             # or 'c:\path\under\windoze'
        -c '%blenderPath/%blenderExe -b %blenderFile %thr -f 1 '
        -b blendfile.blen                 # HAS TO BE!
        -a 0.01
        -i 3
        -t 0
\end{verbatim}
\subsection*{AUTHOR\label{AUTHOR}\index{AUTHOR}}
\begin{verbatim}
        Piotr Arlukowicz,
        Blender Foundation Certified Trainer
        <piotao@polskikursblendera.pl>
        Twitter: /piotao
        Blender Network: /piotr-arlukowicz
\end{verbatim}
\subsection*{BUGS\label{BUGS}\index{BUGS}}


If there are some bugs in this code, then sorry. Nobody is perfect :)
However, if you are unable to repair this script by yourself, you can try to
bother me using my email. I do not promise to help, but you can always try.

\subsection*{DISCLAIMER\label{DISCLAIMER}\index{DISCLAIMER}}


Hereby, I do not have any responsibility and will not take it if you were
pissed off, harmed, killed or injured in any way using this script. Remember,
this script is a one-day project made just for fun, and for testing blender
on a new computer, so it is not intended to be bulletproof. It is not a
rocket-science either. So, try to not shoot yourself in the foot, please :)

\printindex

\end{document}
